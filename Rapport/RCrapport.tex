\documentclass[12pt,a4paper,oneside]{book}

\makeatletter
\newcommand\thefontsize[1]{{}}
\makeatother
\usepackage[utf8]{inputenc}
\usepackage{enumitem}
\usepackage{varwidth}
\usepackage{graphicx}
\usepackage{caption}

\usepackage[top=2.5cm, bottom=3cm, left=2.5cm, right=2.5cm]{geometry}
\usepackage[utf8]{inputenc}
\usepackage[titletoc,title]{appendix}
\usepackage[linewidth=1pt]{mdframed}
\usepackage{framed}
\usepackage{listings}
\usepackage{smartdiagram}
\usepackage{smartdiagram}
\usepackage{varwidth}
\usepackage{amsmath}
\usesmartdiagramlibrary{additions}
\lstdefinestyle{customc}{
	belowcaptionskip=1\baselineskip,
	breaklines=true,
	frame=L,
	xleftmargin=\parindent,
	language=C,
	showstringspaces=false,
	basicstyle=\footnotesize\ttfamily,
	keywordstyle=\bfseries\color{green!40!black},
	commentstyle=\itshape\color{purple!40!black},
	identifierstyle=\color{blue},
	stringstyle=\color{orange},
}

\lstdefinestyle{customasm}{
	belowcaptionskip=1\baselineskip,
	frame=L,
	xleftmargin=\parindent,
	language=[x86masm]Assembler,
	basicstyle=\footnotesize\ttfamily,
	commentstyle=\itshape\color{purple!40!black},
}

\lstset{escapechar=@,style=customc}

\lstset{
	literate=%
	{à}{{\'a}}1
	{í}{{\'i}}1
	{é}{{\'e}}1
	{è}{{\`e}}1
	{ý}{{\'y}}1
	{ú}{{\'u}}1
	{ó}{{\'o}}1
	{ě}{{\v{e}}}1
	{š}{{\v{s}}}1
	{č}{{\v{c}}}1
	{ř}{{\v{r}}}1
	{ž}{{\v{z}}}1
	{ď}{{\v{d}}}1
	{ť}{{\v{t}}}1
	{ň}{{\v{n}}}1
	{ů}{{\r{u}}}1
	{Á}{{\'A}}1
	{Í}{{\'I}}1
	{É}{{\'E}}1
	{Ý}{{\'Y}}1
	{Ú}{{\'U}}1
	{Ó}{{\'O}}1
	{Ě}{{\v{E}}}1
	{Š}{{\v{S}}}1
	{Č}{{\v{C}}}1
	{Ř}{{\v{R}}}1
	{Ž}{{\v{Z}}}1
	{Ď}{{\v{D}}}1
	{Ť}{{\v{T}}}1
	{Ň}{{\v{N}}}1
	{Ů}{{\r{U}}}1
}

\begin{document}

	\tableofcontents

	\listoffigures

	
	\chapter{Introduction }
	bla bla for tp and le but of ir
	\chapter{Réalisation}
	\section{L'outil cygwin}
	  why did we use it and all
	\section{Étape 1:}
	\subsection{Utilisation du prod1vid.m}
	 principalement prod1vid construit un réseau causal probabiliste basé sur le produit tel que les connexions entre les nœuds sont aléatoires , ainsi que les valeurs initiales attribués à la variable d'intérêt et l'évidence .
	 Pour exécuter le programme il faut:
	 \begin{itemize}
	 	\item sur Matlab taper : prod1vid
	 	
	 \end{itemize}
	 ce que le programme offre en sortie est environnement ou on peut voir toutes les variables et le graphe (matrice) crée.
	 on peut alors afficher :
	 
	 \begin{itemize}
	 	\item la variable d'intérêt sachant l'évidence
	 	\item temps de la propagation
	 	\item type de graphe (multi-connected (multi-connectés) ou polytree (polyarbre))
	 \end{itemize}
	 \subsubsection{Fonctionnement du programme}
	 Aprés avoir étudier le programme on a pu résumer son fonctionnement dans les étapes qui suivent :
	 \begin{enumerate}
	 	\item Initialisation du nombre de parents max globale et nombre de noeuds du graphe à construire
	 	\item Création de liens de façon aléatoire entre les noeuds.
	 	\item Utilisation de processus de fixation après la création aléatoire afin d'éviter les noeuds isolés et sous graphes isolés ( les inconvénients de l'aléatoire)
	 	\item Prise de considération des domaines des variables (représentés par les noeuds)
	 	cas binaire etc ...
	 	\item Génération de la distribution aléatoire initiale du graphe crée (de possibilité initiales).
	 	\item génération aléatoire d'une évidence  : une évidence est une information nouvelle qui viens et à qui on aimerait calculer l'influence qu'elle aura sur la variable d'intérêt(évidente est comme une condition ).
	 	\item Détermination si le graphe est polytree ou multi-connected
	 	\item Lancement de la propagation 
	 	
	 \end{enumerate}
	 
	\subsection{Réglage de paramètres}
	\subsubsection{Jeu de test}
	different nb node and  different nbparent max testé manuellement
	\subsubsection{Affichage}
	affichage du temps de propagation each time à chaque execution du prog et du degré de possibilté (basé sur le produit) de la variable d'interet 
	\subsubsection{Explication et observation}
	une explication  sur ce que t'as compris or any observation ( ça serait bien de parler comment fait l'algo generalement et commenter tes resultat)
	
	\section{Etape 2}
	\subsection{Explication + observation }
       i dunno what u can put here
       
    \section{Etape 3}
    \subsection{Géneration automatique aléatoire des Graphes}
    \subsubsection{Géneration des Polytree(youpii)}
    \subsubsection{Géneration des Multiconnected}
    \subsubsection{Géneration des simplement connected}
    \section{Résultats}
    
    \section{Observation pour chaque type }
    \section{Observation et comparaison entre les differents types}

\end{document}